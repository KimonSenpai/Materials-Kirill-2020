\documentclass[a4paper,12pt]{article}
\usepackage{amsmath,amssymb,amsfonts,amsthm}
\usepackage{tikz}
\usepackage [utf8] {inputenc}
\usepackage [T2A] {fontenc} 
\usepackage[russian]{babel}
% Так ссылки в PDF будут активны
\usepackage[unicode]{hyperref}
\usepackage{ textcomp }
\usepackage{indentfirst}


%===============================================================================

% Title settings for Russian language in amsart

%===============================================================================

\makeatletter
\def\@settitle{\begin{center}%
		\baselineskip14\p@\relax
		\bfseries\scshape
		\@title
	\end{center}%
}
\makeatother

%===============================================================================

% Theorem styles

%===============================================================================
\usepackage{amsxtra}
\usepackage{enumitem, fancyref, theoremref}
\usepackage{amsrefs, latexsym, pstricks, mathtext}
\usepackage{hyperref, caption, fancyhdr, listings}

\renewenvironment{proof}{{\bfseries\itshape Доказательство.}}{\begin{flushright}
		$\Box$
\end{flushright}}

\newenvironment{sol}{{\bfseries\itshape Решение.}}{\begin{flushright}
		$\Box$
\end{flushright}}

\theoremstyle{plain}
\newtheorem*{quest}{Вопрос}
\newtheorem{prop}{Предложение}
\newtheorem{theorem}{Теорема}[section]
\newtheorem{lemma}{Лемма}[section]

\theoremstyle{remark}
\newtheorem{conseq}{Вывод}
\newtheorem{rem}{Замечание}

\theoremstyle{definition}
\newtheorem{definition}{Определение}

\newtheoremstyle{problem}
{\topsep}   % ABOVESPACE
{\topsep}   % BELOWSPACE
{\normalfont}  % BODYFONT
{0pt}       % INDENT
{\bfseries} % HEADFONT
{.}         % HEADPUNCT
{5pt plus 1pt minus 1pt} % HEADSPACE
{}          % CUSTOM-HEAD-SPEC
\newtheorem{problem}{Задача}

%===============================================================================



% вы сможете вставлять картинки командой \includegraphics[width=0.7\textwidth]{ИМЯ ФАЙЛА}
% получается подключать, как минимум, файлы .pdf, .jpg, .png.
\usepackage{graphicx}
% Если вы хотите явно указать поля:
\usepackage[margin=1in]{geometry}
% Или если вы хотите задать поля менее явно (чем больше DIV, тем больше места под текст):
% \usepackage[DIV=10]{typearea}

\usepackage{fancyhdr}

\newcommand{\bbR}{\mathbb R}%теперь вместо длинной команды \mathbb R (множество вещественных чисел) можно писать короткую запись \bbR. Вместо \bbR вы можете вписать любую строчку букв, которая начинается с '\'.
\newcommand{\eps}{\varepsilon}
\newcommand{\bbN}{\mathbb N}
\newcommand{\dif}{\mathrm{d}}

\newtheorem{Def}{Definition}


\pagestyle{fancy}
\makeatletter % сделать "@" "буквой", а не "спецсимволом" - можно использовать "служебные" команды, содержащие @ в названии
\fancyhead[L]{\footnotesize \@title}%Это будет написано вверху страницы слева
\fancyhead[R]{\footnotesize <<Физтех-лицей>> им. П. Л. Капицы}
\fancyfoot[L]{\footnotesize \@author}%имя автора будет написано внизу страницы слева
\fancyfoot[R]{\@date}%номер страницы —- внизу справа
\fancyfoot[C]{\thepage}%по центру внизу страницы пусто

\renewcommand{\maketitle}{%Настройка заголовка
	\noindent{\bfseries\scshape\large\@title\ \mdseries\upshape}\par
	\noindent {\large\itshape Author: \@author}
	\vskip 2ex}
\makeatother
\def\dd#1#2{\frac{\partial#1}{\partial#2}}
\def\ssum#1#2{\sum\limits_{#1}^{#2}}
\def\l{\langle}
\def\r{\rangle}

\graphicspath{{images//}}

\hypersetup{
	pdfauthor={Kim Zyong}
}

\usepackage{multirow}
\usepackage{colortbl}

\title{Задание тригонометрия}
\author{Ким Зыонг} 
\date{\today}

\begin{document}
	\maketitle
	\begin{problem}
		Из колоды, содержащей 52 карты, вынули 10 карт. Сколько существует способов вынуть их так, чтобы среди этих карт
		\begin{enumerate}
			\item  был хотя бы один туз;
			
			\item  был ровно один туз;
			
			\item  было не менее двух тузов;
			
			\item  было ровно два туза?
		\end{enumerate}
	\end{problem}

	\begin{problem}
		За круглым столом сидят $n$ рыцарей. Сколькими способами можно из них отобрать $k$ рыцарей, чтобы в их число не попали никакие два сидящих рядом?
	\end{problem}

	\begin{problem}
		Из 100 человек 85 знают английский язык, 80 -- испанский, 75 -- немецкий. Сколько человек заведомо знают все три языка?
		
	\end{problem}

	\begin{problem}
		Сколько существует чисел, меньших 1000, которые не кратны ни 2, ни 3, ни 5?
	\end{problem}

	\begin{problem}
		На книжной полке стоят $n$ книг. Сколькими способами можно выбрать из них $k$ книг так, чтобы в их число не попали никакие две стоящие рядом?
	\end{problem}

	\begin{problem}
		Бросают $n$ игральных костей. В результате получают $n$ чисел от 1 до 6 Сколько может получиться различных результатов, если результаты, отличающиеся друг от друга лишь порядком очков, считаются одинаковыми?
	\end{problem}

	\begin{problem}
		Сколько натуральных делителей имеет число $2007^{2007}$?
	\end{problem}

	\begin{problem}
		Каждая сторона квадрата разбита на $n$ частей. Сколько можно построить треугольников, вершинами которых являются точки разбиения (вершины квадрата такими точками не считать)?
	\end{problem}

	\begin{problem}
		Сколькими способами можно разделить $n_1$ предметов первого вида, $n_2$ предметов второго вида, ..., $n_k$ предметов $k$-го вида по $m$ различным ящикам, считая, что предметы каждого вида неразличимы между собой?
	\end{problem}
\end{document}